\section{Generating diff files}\label{sec:diffFiles}
\begin{enumerate}
\item If you want to generate diff file for two folders (e.g., one folder has updated version and one folder has the old version), you can use the diff.sh in this folder to do so. It requires three inputs: old folder, new folder and output folder. The script loops all tex files in the sections folder and root folder to compute the diff between the file in the older folder and the one in the new folder with the same file name and output the file into the output folder. Besides, the script copies files in other folders (e.g., image, figure\_scripts) to the output folder. After the execution, you can compile files in the output folder to get diff file. Example: \verb|bash diff.sh submitted_version new_version diff|

\item Instead, if you want to generate diff file between two commits on Github, things become easier.
\end{enumerate}
\begin{itemize}
\item Install latexdiff first. Check instructions for Ubuntu, Windows and MacOS.
\item Install git-latexdiff. Check instructions here.
\item Run the following commands based on your needs: Diff the previous revision with the latest commit:
\end{itemize}

\verb|git latexdiff HEAD^ --main main.tex|

Diff the latest commit with the working tree:

\verb|git latexdiff HEAD -- --main main.tex|

Diff latest commit with branch master:

\verb|gt latexdiff master HEAD --main main.tex|

Pass --type=CHANGEBAR to latexdiff to get changebars in the margins instead of red+trike/blue+underline diff:

\verb|git latexdiff --type=CHANGEBAR HEAD^ --main main.tex|

Use a specific latexdiff configuration file:

\verb|git latexdiff --config /path/to/file HEAD^ --main main.tex|

Note that, for those two versions, you should set compileFigures to 0. The diff will not highlight the changes of figures inside the tikzpicture but will highlight changes in the captions. In addition, you need to generate bbl manually (by runing bibtex main.tex) for these two versions and \verb|run git latexdiff HEAD^ --main main.tex ----flatten| if you want to see the changes in the reference.bib.