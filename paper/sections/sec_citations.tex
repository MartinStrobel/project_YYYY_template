\section{Citations}\label{sec:citations}

When you want to create a new reference. Use the bibtex entries provided by Google Scholar.

\begin{enumerate}
	\item Go to \href{https://scholar.google.com/}{Google Scholar}.
\item Search for the paper.
\item Go on All <versions> and find the right version (i.e. where possible the final conference/journal version)
\item Click on cite ()
\item Click on BibTeX and copy the entry
\item Double-check and fix the entry if something is wrong.
\end{enumerate}

If you have to create a bibtex entry from scratch (because it doesn't exist on Google Scholar), follow the reference format \verb|<lastname_first_author><yyyy><first_proper_word_title>|.

\begin{whyblock}
 Everyone following one convention avoids duplicate citations. GoogleScholar has an easy format and is a good start for almost any bibtex entry.
\end{whyblock}

\textbf{Optional:} If you want consistent references to a venue you can use the abb.bib file. Change the booktitle to \verb|booktitle=proc # {<xx>th} # <venue_name>|

\textbf{Tip:} There is a Google Scholar browser extension that creates a button to directly search.